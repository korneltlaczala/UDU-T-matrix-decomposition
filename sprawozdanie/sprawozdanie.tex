\documentclass{article}

\usepackage[polish]{babel}
\usepackage[utf8]{inputenc}
\usepackage[T1]{fontenc}
\usepackage{amsmath, amssymb}

\usepackage[letterpaper,top=2cm,bottom=2cm,left=3cm,right=3cm,marginparwidth=1.75cm]{geometry}
\usepackage{graphicx}

\usepackage{amsmath}
\usepackage{graphicx}
\usepackage[colorlinks=true, allcolors=blue]{hyperref}

\title{Rozwiązywanie układu równań liniowych \(Ax = b\), gdzie \(A(n \times n)\) jest macierzą symetryczną dodatnio określoną}
\author{Kornel Tłaczała}

\begin{document}
\maketitle

\vspace{0.2cm}
\begin{center}
\textbf{\large Projekt nr 2} \\
\end{center}

\section{Opis probelmu}
    \subsection*{Cel projektu}
Niech \(A \in \mathbb{R}^{n \times n} \) będzie dodatnio określoną macierzą symetryczną postaci:
\[
A=\begin{bmatrix}
    A_{11} & A_{12}\\
    A_{12}^T & A_{22}
\end{bmatrix},
\quad A_{ij} \in \mathbb{R}^{p \times p} \quad \wedge \quad n=2p
\]
Zadaniem jest rozwiązać układ równań liniowych \(Ax = b\) korzystając z blokowego rozkładu \(UDU^T\) macierzy \(A\) (\(A = UDU^T\)).
    \subsection*{Metoda rozwiązania}
Jeżeli przyjmiemy, że U jest macierzą o postaci:
\[
U=\begin{bmatrix}
    I & U_{12}\\
    0 & I
\end{bmatrix}
\]
oraz D jest macierzą o postaci:
\[
D=\begin{bmatrix}
    D_{11} & 0\\
    0 & D_{22}
\end{bmatrix}
\]
To otrzymujemy:

\[
A = UDU^T
\]
\vspace{5pt}
\[
\begin{bmatrix}
    A_{11} & A_{12}\\
    A_{12}^T & A_{22}
\end{bmatrix}
=
\begin{bmatrix}
    I & U_{12}\\
    0 & I
\end{bmatrix}
\cdot
\begin{bmatrix}
    D_{11} & 0\\
    0 & D_{22}
\end{bmatrix}
\cdot
\begin{bmatrix}
    I & 0\\
    U_{12}^T & I
\end{bmatrix}
=
\]
\[
=
\begin{bmatrix}
    D_{11} & U_{12} \cdot D_{22}\\
    0 & D_{22}
\end{bmatrix}
\cdot
\begin{bmatrix}
    I & 0\\
    U_{12}^T & I
\end{bmatrix}
=
\]
\[
=
\begin{bmatrix}
    D_{11} + U_{12} \cdot D_{22} \cdot U_{12}^T & U_{12} \cdot D_{22}\\
    D_{22} \cdot U_{12}^T & D_{22}
\end{bmatrix}
\]


\vspace{10pt}

\end{document}